% !TEX root = ../../paper.tex

\section{Code and Math} % (fold)
\label{sec:code_and_math}

\subsection{Code Blocks} % (fold)
\label{sub:code_blocks}

Code blocks look like this:

	\begin{listing}[H]
	\begin{minted}{c}
float Q_rsqrt( float number )
{
	long i;
	float x2, y;
	const float threehalfs = 1.5F;

	x2 = number * 0.5F;
	y  = number;
	i  = * ( long * ) &y;                       // evil floating point bit level hacking
	i  = 0x5f3759df - ( i >> 1 );               // what the f@*#? 
	y  = * ( float * ) &i;
	y  = y * ( threehalfs - ( x2 * y * y ) );   // 1st iteration
//	y  = y * ( threehalfs - ( x2 * y * y ) );   // 2nd iteration, this can be removed

	return y;
}
	\end{minted}
	\caption{Fast Inverse Square-Root operation from Quake III Arena \parencite{quake}}
	\end{listing}
% subsection code_blocks (end)

\subsection{Inline code} % (fold)
\label{sub:inline_code}
Inline code looks like this: \mintinline{c}{const double pi = 3.1415926535}
% subsection inline_code (end)

\subsection{Math} % (fold)
\label{sub:math}
This is a cool latex section writing cool stuff \parencite{article}. Here's some math, demonstrating \href{https://en.wikipedia.org/wiki/Information_theory#Entropy_of_an_information_source}{Shannon's entropy of an information source}.

\begin{equation}
\begin{split}
H(X) = \mathbb{E}_X [I(x)] = -\sum_{x \in \mathbb{X}}p(x)\log p(x)
\end{split}
\end{equation}

What about some inline math demonstrating \href{https://en.wikipedia.org/wiki/Euler%27s_identity}{Euler's identity}? $e^{i\pi} + 1 = 0$
% subsection math (end)

% section code_and_math (end)
